%% abtex2-modelo-trabalho-academico.tex, v-1.9.6 laurocesar
%% Copyright 2012-2016 by abnTeX2 group at http://www.abntex.net.br/ 
%%
%% Modificado por Marcus Nunes para se adequar
%% às exigências do Departamento de Estatística
%% da Universidade Federal do Rio Grande do Norte
%% 
%% http://marcusnunes.me
%%
%% This work may be distributed and/or modified under the
%% conditions of the LaTeX Project Public License, either version 1.3
%% of this license or (at your option) any later version.
%% The latest version of this license is in
%%   http://www.latex-project.org/lppl.txt
%% and version 1.3 or later is part of all distributions of LaTeX
%% version 2005/12/01 or later.
%%
%% This work has the LPPL maintenance status `maintained'.
%% 
%% The Current Maintainer of this work is the abnTeX2 team, led
%% by Lauro César Araujo. Further information are available on 
%% http://www.abntex.net.br/
%%
%% This work consists of the files abntex2-modelo-trabalho-academico.tex,
%% abntex2-modelo-include-comandos and abntex2-modelo-references.bib
%%

% ------------------------------------------------------------------------
% ------------------------------------------------------------------------
% abnTeX2: Modelo de Trabalho Academico (tese de doutorado, dissertacao de
% mestrado e trabalhos monograficos em geral) em conformidade com 
% ABNT NBR 14724:2011: Informacao e documentacao - Trabalhos academicos -
% Apresentacao
% ------------------------------------------------------------------------
% ------------------------------------------------------------------------

\documentclass[
	% -- opções da classe memoir --
	12pt,				% tamanho da fonte
	openright,			% capítulos começam em pág ímpar (insere página vazia caso preciso)
	oneside,			% para impressão em recto e verso. Oposto a oneside
	a4paper,			% tamanho do papel. 
	% -- opções da classe abntex2 --
	%chapter=TITLE,		% títulos de capítulos convertidos em letras maiúsculas
	%section=TITLE,		% títulos de seções convertidos em letras maiúsculas
	%subsection=TITLE,	% títulos de subseções convertidos em letras maiúsculas
	%subsubsection=TITLE,% títulos de subsubseções convertidos em letras maiúsculas
	% -- opções do pacote babel --
	english,			% idioma adicional para hifenização
	french,				% idioma adicional para hifenização
	spanish,			% idioma adicional para hifenização
	brazil				% o último idioma é o principal do documento
	]{abntex2}

% ---
% Pacotes básicos 
% ---
\usepackage{multirow,lscape,array}
\usepackage{amsmath,amsthm,amssymb}
\usepackage{lmodern}			% Usa a fonte Latin Modern			
\usepackage[T1]{fontenc}		% Selecao de codigos de fonte.
\usepackage[utf8]{inputenc}		% Codificacao do documento (conversão automática dos acentos)
\usepackage{lastpage}			% Usado pela Ficha catalográfica
\usepackage{indentfirst}		% Indenta o primeiro parágrafo de cada seção.
\usepackage{color}				% Controle das cores
\usepackage{graphicx}			% Inclusão de gráficos
\usepackage{graphics}			% Inclusão de gráficos
\usepackage{amssymb}            % Inclusão de símbolos
\usepackage{microtype} 			% para melhorias de justificação
\usepackage{icomma}
\usepackage[final]{pdfpages}

% ---
		
% ---
% Pacotes adicionais, usados apenas no âmbito do Modelo Canônico do abnteX2
% ---
\usepackage{lipsum}				% para geração de dummy text
% ---

% ---
% Pacotes de citações
% ---
%\usepackage[brazilian,hyperpageref]{backref}	 % Paginas com as citações na bibl
\usepackage[alf]{abntex2cite}	% Citações padrão ABNT

% --- 
% CONFIGURAÇÕES DE PACOTES
% --- 

% ---
% Configurações do pacote backref
% Usado sem a opção hyperpageref de backref
%\renewcommand{\backrefpagesname}{Citado na(s) página(s):~}
% Texto padrão antes do número das páginas
%\renewcommand{\backref}{}
% Define os textos da citação
%\renewcommand*{\backrefalt}[4]{
%	\ifcase #1 %
%		Nenhuma citação no texto.%
%	\or
%		Citado na página #2.%
%%		Citado #1 vezes nas páginas #2.%
	%\fi}%
% ---

% ---
% Informações de dados para CAPA e FOLHA DE ROSTO
% ---
\titulo{Título da Monografia}
\autor{Nome do Discente}
\local{Natal - RN}
\data{7 de dezembro de 2017}
\orientador{Prof. Dr. Marcus Alexandre Nunes}
%\coorientador{Equipe \abnTeX}
\instituicao{%
  Universidade Federal do Rio Grande do Norte
  \par 
  Centro de Ciências Exatas e da Terra
  \par
  Departamento de Estatística
  \par
  }
  %Programa de Pós-Graduação}
\tipotrabalho{Monografia (Graduação)}
% O preambulo deve conter o tipo do trabalho, o objetivo, 
% o nome da instituição e a área de concentração 
\preambulo{Monografia de Graduação apresentada ao Departamento de Estatística do Centro de Ciências Exatas e da Terra da Universidade Federal do Rio Grande do Norte como requisito parcial para a obtenção do grau de Bacharel em Estatística.}
% ---


% ---
% Configurações de aparência do PDF final

% alterando o aspecto da cor azul
\definecolor{blue}{RGB}{41,5,195}

% informações do PDF
\makeatletter
\hypersetup{
     	%pagebackref=true,
		pdftitle={\@title}, 
		pdfauthor={\@author},
    	pdfsubject={\imprimirpreambulo},
	    pdfcreator={LaTeX with abnTeX2},
		pdfkeywords={Má especificação}{Estatísticas Robustas}{Família de Posição e Escala}{Modelos de Regressão}{Estatística Gradiente}, 
		bookmarksdepth=4
}
\makeatother
% --- 

% --- 
% Espaçamentos entre linhas e parágrafos 
% --- 

% O tamanho do parágrafo é dado por:
\setlength{\parindent}{1.3cm}

% Controle do espaçamento entre um parágrafo e outro:
\setlength{\parskip}{0.2cm}  % tente também \onelineskip

\theoremstyle{plain}
\newtheorem{theorem}{Teorema}[section]
\newtheorem{axiom}{Axioma}[section]
\newtheorem{corollary}{Corolário}[section]
\newtheorem{lemma}{Lema}[section]
\newtheorem{proposition}{Proposição}[section]
%-----------------------------------------------------------
\theoremstyle{definition}
\newtheorem{definition}{Definição}[section]
\newtheorem{example}{Exemplo}[section]
%-----------------------------------------------------------
\theoremstyle{remark}
\newtheorem{remark}{Observação}[section]

% ---
% compila o indice
% ---
\makeindex
% ---
\newcommand{\R}{\mathbb{R}}
\newcommand{\N}{\mathbb{N}}
\newcommand{\Z}{\mathbb{Z}}
\newcommand{\Q}{\mathbb{Q}}

\newcommand{\marcus}[2]{\textcolor{red}{#1}\footnote{Marcus: #2}\xspace}

% ----
% Início do documento
% ----
\begin{document}

% Seleciona o idioma do documento (conforme pacotes do babel)
%\selectlanguage{english}
\selectlanguage{brazil}

% Retira espaço extra obsoleto entre as frases.
\frenchspacing 


% ----------------------------------------------------------
% ELEMENTOS PRÉ-TEXTUAIS
% ----------------------------------------------------------
% \pretextual

% ---
% Capa
% ---
\imprimircapa
% ---

% ---
% Folha de rosto
% (o * indica que haverá a ficha bibliográfica)
% ---
\imprimirfolhaderosto*

% ---

% ---
% Inserir a ficha bibliografica
% ---

% Isto é um exemplo de Ficha Catalográfica, ou ``Dados internacionais de
% catalogação-na-publicação''. Você pode utilizar este modelo como referência. 
% Porém, provavelmente a biblioteca da sua universidade lhe fornecerá um PDF
% com a ficha catalográfica definitiva após a defesa do trabalho. Quando estiver
% com o documento, salve-o como PDF no diretório do seu projeto e substitua todo
% o conteúdo de implementação deste arquivo pelo comando abaixo:
%
 \begin{fichacatalografica}
 	% editar a linha abaixo com a ficha catalografica
 	% fornecida pela biblioteca
    %\includepdf{FichaCatalografica.pdf}
 \end{fichacatalografica}



% ---
% Inserir folha de aprovação
% ---

% Isto é um exemplo de Folha de aprovação, elemento obrigatório da NBR
% 14724/2011 (seção 4.2.1.3). Você pode utilizar este modelo até a aprovação
% do trabalho. Após isso, substitua todo o conteúdo deste arquivo por uma
% imagem da página assinada pela banca com o comando abaixo:
%
%
% editar a linha abaixo com a folha de aprovacao
% fornecida pela biblioteca
%\includepdf{FolhadeRos.pdf}
\begin{folhadeaprovacao}

  \begin{center}
    {\ABNTEXchapterfont\large\imprimirautor}

    \vspace*{\fill}\vspace*{\fill}
    \begin{center}
      \ABNTEXchapterfont\bfseries\Large\imprimirtitulo
    \end{center}
    \vspace*{\fill}
    
    \hspace{.45\textwidth}
    \begin{minipage}{.5\textwidth}
        \imprimirpreambulo
    \end{minipage}%
    \vspace*{\fill}
   \end{center}
        
   Aprovado em \hphantom{XXX} de \hphantom{XXXXXXXXXXX} de \hphantom{XXXXXX}.

   \assinatura{\textbf{\imprimirorientador} \\ Orientador} 
   \assinatura{\textbf{Profª. Drª. Fulana} \\ Examinadora}
   \assinatura{\textbf{Prof. Dr. Beltrano} \\ Examinador}
   %\assinatura{\textbf{Professor} \\ Convidado 3}
   %\assinatura{\textbf{Professor} \\ Convidado 4}
      
   \begin{center}
    \vspace*{0.5cm}
    {\large\imprimirlocal}
    \par
    {\large\imprimirdata}
    \vspace*{1cm}
  \end{center}
  
\end{folhadeaprovacao}
% ---

% ---
% Dedicatória
% ---
\begin{dedicatoria}
   \vspace*{\fill}
   \flushright
   \noindent
   \textit{Alguém importante.} \vspace*{3cm}
\end{dedicatoria}
% ---

% ---
% Agradecimentos
% ---
\begin{agradecimentos}

A todo mundo que é importante.

\end{agradecimentos}
% ---

% ---
% Epígrafe
% ---
\begin{epigrafe}
    \vspace*{\fill}
	\begin{flushright}
		\textit{``Uma frase legal.'' \\
		Autor da frase legal} \vspace*{3cm}
	\end{flushright}
\end{epigrafe}
% ---

% ---
% RESUMOS
% ---

% resumo em português
\setlength{\absparsep}{18pt} % ajusta o espaçamento dos parágrafos do resumo
\begin{resumo}
Resumo da monografia. Escreva um texto no qual o tema do seu trabalho seja explicado. Explicite a importância do tema e do problema da sua pesquisa. Pensando nisso, reporte o que foi feito no trabalho para produzi-lo. Descreva as ferramentas utilizadas para chegar no seu objetivo. Não esqueça de descrever quais foram os resultados mais importantes e, ao final, o que é possível concluir com seu trabalho.

 \textbf{Palavras-chave}: Palavras. Chave. Separadas por Ponto.
\end{resumo}

% resumo em inglês
\begin{resumo}[Abstract]
 \begin{otherlanguage*}{english}
Translate your ``resumo'' to English.

   \vspace{\onelineskip}
 
   \noindent 
   \textbf{Keywords}: Keywords. Dot separated.
 \end{otherlanguage*}
\end{resumo}
% ---


% ---
% inserir lista de ilustrações
% ---
\pdfbookmark[0]{\listfigurename}{lof}
\listoffigures*
\cleardoublepage
% ---


% ---
% inserir lista de tabelas
% ---% ---
\pdfbookmark[0]{\listtablename}{loc}
\listoftables
\cleardoublepage

% ---
% inserir lista de abreviaturas e siglas
% ---
% inserir lista de símbol
% ---
% inserir o sumario
% ---
\pdfbookmark[0]{\contentsname}{toc}
\tableofcontents* 
\cleardoublepage
% ---



% ----------------------------------------------------------
% ELEMENTOS TEXTUAIS
% ----------------------------------------------------------
\textual


\renewcommand{\thetable}{\thechapter.\arabic{table}}
\setcounter{table}{0}
\renewcommand{\thefigure}{\thechapter.\arabic{figure}}
\setcounter{figure}{0}
\renewcommand{\theequation}{\thechapter.\arabic{equation}}
\setcounter{equation}{0}

\chapter{Introdução}\label{introducao}

Utilize este espaço para escrever a introdução da sua monografia. Ela deve ter as seguintes características:

\begin{itemize}

  \item Tema bem delimitado, apresentando qual a função do seu trabalho
  \item O texto deve ser breve
  \item Apresente o problema da pesquisa da sua monografia
  \item Descreva a relevância do tema escolhido e do trabalho realizado
  \item Apresente os objetivos geral (ideia central) e específicos (resultados a serem atingidos)
  \item De forma sucinta, comente sobre os instrumentos e métodos utilizados para desenvolver seu trabalho
  \item Escreva um parágrafo como este abaixo, com a estrutura dos capítulos

\end{itemize}

A revisão bibliográfica e nossa motivação é feita no Capítulo \ref{revisaobibliografica}. Descrevemos o modelo estudado neste trabalho no Capítulo \ref{modelagem}. O Capítulo \ref{resultados} trata dos resultados dos estudos realizados. A conclusão é feita no Capítulo \ref{consideracoesfinais}.

Os capítulos listados acima são apenas uma sugestão de organização da monografia. A quantidade de capítulos pode ser aumentada ou diminuída de acordo com a preferência do aluno e do orientador.



\renewcommand{\thetable}{\thechapter.\arabic{table}}
\setcounter{table}{0}
\renewcommand{\thefigure}{\thechapter.\arabic{figure}}
\setcounter{figure}{0}
\renewcommand{\theequation}{\thechapter.\arabic{equation}}
\setcounter{equation}{0}

\chapter{Revisão Bibliográfica}\label{revisaobibliografica}

Aqui listamos os artigos e livros importantes para este trabalho. Por exemplo, os livros \citeonline{Hilbe2011} e \citeonline{Nelder1989} são fundamentais para o trabalho realizado. Além disso, estudamos como clusterizar dados longitudinais através do método KmL \cite{Genolini2009}.



\renewcommand{\thetable}{\thechapter.\arabic{table}}
\setcounter{table}{0}
\renewcommand{\thefigure}{\thechapter.\arabic{figure}}
\setcounter{figure}{0}
\renewcommand{\theequation}{\thechapter.\arabic{equation}}
\setcounter{equation}{0}

\chapter{Modelagem}\label{modelagem}

Neste capítulo descrevemos o modelo utilizado neste trabalho. Como estamos supondo que existe relação linear entre as variáveis $x$ e $y$, utilizamos o modelo de regressão linear dado pela equação \eqref{ModeloLinear}:

\begin{equation}\label{ModeloLinear}
y_i = \beta_0 + \beta_1x_i + \varepsilon_i
\end{equation}


\renewcommand{\thetable}{\thechapter.\arabic{table}}
\setcounter{table}{0}
\renewcommand{\thefigure}{\thechapter.\arabic{figure}}
\setcounter{figure}{0}
\renewcommand{\theequation}{\thechapter.\arabic{equation}}
\setcounter{equation}{0}

\chapter{Resultados}\label{resultados}

Os resultados vem aqui.

\section{Simulação}\label{resultadosSimulacao}

É possível criar seções e subseções dentro do documento. Isso vai deixar o trabalho mais organizado.




\subsection{Simulação Particular}\label{resultadosSimulacao_sub}

Escreva neste local resultados particulares de alguma simulação realizada, caso isso seja necessário.



\section{Aplicação}\label{aplicacao}

Aplicação a dados reais do método descrito no Capítulo \ref{modelagem}. A Figura \ref{serietemporal} mostra os dados analisados neste trabalho. 


\begin{figure}[!h]
\centering
\includegraphics[width=0.8\textwidth]{figuras/serie_temporal.pdf}
\label{serietemporal}
\end{figure}


Devido a características do próprio LaTeX, as figuras podem acabar ficando em posições estranhas às vezes. Em geral, quanto mais texto for escrito, mais fácil é para o programa encontrar locais mais adequados para as figuras.


\marcus{Este comando é interessante. Ele está definido na linha 186 do arquivo 00-Monografia.tex.}{Com este comando, é possível o orientador fazer comentários mais efetivos na correção do texto do aluno, caso ambos estejam usando o Overleaf para trabalhar.}


\renewcommand{\thetable}{\thechapter.\arabic{table}}
\setcounter{table}{0}
\renewcommand{\thefigure}{\thechapter.\arabic{figure}}
\setcounter{figure}{0}
\renewcommand{\theequation}{\thechapter.\arabic{equation}}
\setcounter{equation}{0}

\chapter{Considerações Finais}\label{consideracoesfinais}

Conclusão da monografia.


\bibliography{bibliografia}



\end{document}
