\chapter{Revisão Bibliográfica}\label{revisaobibliografica}

Aqui listamos os artigos e livros importantes para este trabalho. É importante procurar referências na literatura, sejam em livros, artigos ou outros trabalhos de conclusão, que fundamentem o seu trabalho. As referências podem trazer informações, conceitos e metodologias a serem utilizadas.

Com as fontes coletadas, analise-as para determinar como elas se relacionam com o seu trabalho. Estas fontes podem ser uma base teórica para o trabalho ou podem ser uma aplicação similar à sua proposta. Elas, inclusive, podem possuir deficiências para as quais o seu trabalho está buscando soluções.

Construa uma estrutura capaz de organizar os seus referenciais teóricos, de modo a ordená-los dentro do recorte do seu trabalho. Crie uma relação entre os trabalhos, relacionando-os entre si e entre os objetivos da monografia.

Em um trabalho sobre regressão utilizando dados de contagem, os livros \citeonline{Hilbe2011} e \citeonline{Nelder1989} são fundamentais para a base teórica dos modelos lineares generalizados que tratam de dados deste tipo. São livros com a parte teórica muito bem desenvolvida. 

Todos os referenciais teóricos devem estar dentro do arquivo \texttt{bibliografia}, no formato utilizado pelo BibTeX.  


