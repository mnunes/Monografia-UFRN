\chapter{Introdução}\label{introducao}

Utilize este espaço para escrever a introdução da sua monografia. Ela deve ter as seguintes características:

\begin{itemize}

  \item Tema bem delimitado, apresentando qual a função do seu trabalho
  \item O texto deve ser breve
  \item Apresente o problema da pesquisa da sua monografia
  \item Descreva a relevância do tema escolhido e do trabalho realizado
  \item Apresente os objetivos geral (ideia central) e específicos (resultados a serem atingidos)
  \item De forma sucinta, comente sobre os instrumentos e métodos utilizados para desenvolver seu trabalho
  \item Escreva um parágrafo como este abaixo, com a estrutura dos capítulos

\end{itemize}

A revisão bibliográfica e nossa motivação é feita no Capítulo \ref{revisaobibliografica}. Descrevemos o modelo estudado neste trabalho no Capítulo \ref{modelagem}. O Capítulo \ref{resultados} trata dos resultados dos estudos realizados. A conclusão é feita no Capítulo \ref{consideracoesfinais}.

Os capítulos listados acima são apenas uma sugestão de organização da monografia. A quantidade de capítulos pode ser aumentada ou diminuída de acordo com a preferência do aluno e do orientador.

